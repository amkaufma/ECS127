\documentclass[paper=letter, fontsize=12pt]{article}
\usepackage[T1]{fontenc}
\usepackage{times}
\usepackage[english]{babel}
\usepackage{sectsty}
\usepackage{textcomp}
\usepackage{amsmath,amsfonts,amsthm}
\usepackage{mathtools}
\usepackage{enumitem}
\usepackage{listings}

\newcommand\tab[1][1cm]{\hspace*{#1}}

\allsectionsfont{\centering \normalfont\scshape}

\newcommand{\horrule}[1]{\rule{\linewidth}{#1}}
\title{
\normalfont \normalsize
\horrule{0.5pt} \\[0.3cm]
\huge Problem Set #1 \\
\horrule{0.5pt} \\[0.3cm]
}

\author{Andrew Kaufman \\ SID: 998048873}
\date{\normalsize\today}

\begin{document}
\maketitle

\section*{Problem 5}
\begin{enumerate}[label=(\alph*)]
\item
\item
\item
\end{enumerate}

\section*{Problem 6}

The following probabilities were obtained after running the RC4 algorithm for 100000 iterations: \\ \\
\tab 0: 0.00815 \\
\tab 1: 0.00386 \\
\tab 2: 0.00393 \\
\tab 3: 0.0036 \\
\tab 4: 0.00416 \\
\tab 5: 0.00393 \\
\tab 6: 0.00417 \\
\tab 7: 0.00374 \\
\tab 8: 0.00368 \\
\tab 9: 0.00384 \\ \\
A simple adversary can attempt to distinguish RC4 output from truly random bits by looking at the second byte of the output
and outputting a 1 if the second byte is 0, or outputting a 0 otherwise. If the adversary returns a 1 the output is interpreted
as true RC4 output The reason we have the adversary doing this is because the probability of the second byte being 0 is much
larger than the probabilities of the other possible values. If the second byte is a 0, it is more probable to be real RC4 output. \\ \\
The advantage of the adversary can then be defined as follows: \\ \\
\tab \tab \textit{Adv(A)} = \big[Pr[A(F) = 1] - Pr[A(G) = 1]\big] \\ \\
where F denotes the algorithm for RC4 output and G denotes the algorithm for random bits. \\ \\
The advantage of the adversary that we described would then be: \\ \\
\tab \tab \textit{Adv(A)} = \(\dfrac{2}{256}\) - \(\dfrac{1}{156}\) = \(\dfrac{1}{256}\) \\ \\
\lstinputlisting{ps3.py}

\section*{Problem 7}

\end{document}
