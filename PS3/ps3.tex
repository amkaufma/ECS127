\documentclass[paper=letter, fontsize=12pt]{article}
\usepackage[T1]{fontenc}
\usepackage{times}
\usepackage[english]{babel}
\usepackage{sectsty}
\usepackage{textcomp}
\usepackage{amsmath,amsfonts,amsthm}
\usepackage{mathtools}
\usepackage{enumitem}
\usepackage{listings}

\newcommand\tab[1][1cm]{\hspace*{#1}}

\allsectionsfont{\centering \normalfont\scshape}

\newcommand{\horrule}[1]{\rule{\linewidth}{#1}}
\title{
\normalfont \normalsize
\horrule{0.5pt} \\[0.3cm]
\huge Problem Set #1 \\
\horrule{0.5pt} \\[0.3cm]
}

\author{Andrew Kaufman \\ SID: 998048873}
\date{\normalsize\today}

\begin{document}
\maketitle

\section*{Problem 5}
\begin{enumerate}[label=(\alph*)]
  \item Alice shuffles a deck of cards and deals it out to herself and Bob so that each gets half of the
  52 cards. Alice now wishes to send a secret message \textit{M} to Bob. Eavesdropper Eve is watching and sees
  the transmissions. \\ \\
  Suppose Alice\textquotesingle s message M \epsilon \(\(\{\)0, 1\(\}\)^\(48\)\) is a string of 48 bits. Describe how Alice can communicate M to
  Bob in a way that achieves perfect privacy. \\ \\


\section*{Problem 6}
\lstinputlisting{ps3.py}

\end{document}
